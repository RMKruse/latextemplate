\begin{titlepage}

    % \if0\blind
    % {
      \title{\bf Model Averaging for Linear Mixed Models via Augmented Lagrangian}
      \author{Ren\'e-Marcel Kruse and Benjamin S\"afken\thanks{
        The authors gratefully acknowledge \textit{please remember to list all relevant funding sources in the unblinded version}}\hspace{.2cm}\\
        Chair of Statistics, University of G\"ottingen\\}
      \maketitle
    % } \fi
    
    % \if1\blind
    % {
    %   \bigskip
    %   \bigskip
    %   \bigskip
    %   \begin{center}
    %     {\LARGE\bf Title}
    % \end{center}
    %   \medskip
    % } \fi
    
    \bigskip
    \begin{abstract}
    Model selection for linear mixed models has been a focus of recent research in statistics. The method of model averaging, however, has been not in the centre of attention. This paper presents a weight finding method for model averaging of linear mixed models, as well as its implementation. However, the optimization of that criterion is a non-trivial task due to its non-linear characteristics and the constraints imposed. Therefore, this paper proposes to use the augmented Lagrangian optimization method which is a non-typical approach for the field of statistics. The characteristics of the approach as well as its implementation are discussed in detail. Additionally, we present a stable, robust and fast implementation of the algorithm, which, together with the weight-finding methodology provided, is released as an extension of the R-package \textbf{cAIC4}. Furthermore, the influence of the weight finding criterion on the resulting model averaging estimator will be discussed through simulation studies as well as an application to real data.
    \\
    \\
    % The text of your abstract.  200 or fewer words.
    \end{abstract}
    
    \noindent%
    {\it Keywords:} Optimization, Augmented Lagrangian, \textbf{cAIC4}, Model Averaging, Linear Mixed Models, conditional AIC
    \\
    % \\ 
    \vfill
    \end{titlepage}